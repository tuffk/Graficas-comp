\documentclass[10pt, a4paper, twocolumn]{article} 
\usepackage[english]{babel}
\usepackage{listings}
\usepackage{microtype}
\usepackage{amsmath,amsfonts,amsthm}
\usepackage[svgnames]{xcolor}
\usepackage[hang, small, labelfont=bf, up, textfont=it]{caption} 
\usepackage{booktabs}
\usepackage{lastpage}
\usepackage{graphicx}
\usepackage{enumitem}
\setlist{noitemsep}
\usepackage{sectsty}
\allsectionsfont{\usefont{OT1}{phv}{b}{n}}
\usepackage{geometry}
\geometry{
	top=1cm,
	bottom=1.5cm, 
	left=2cm,
	right=2cm,
	includehead, 
	includefoot
}
\setlength{\columnsep}{7mm}
\usepackage[T1]{fontenc} 
\usepackage[utf8]{inputenc} 
\usepackage{XCharter}
\usepackage{fancyhdr} 
\pagestyle{fancy} 
\renewcommand{\headrulewidth}{0.0pt}
\renewcommand{\footrulewidth}{0.4pt} 
\renewcommand{\sectionmark}[1]{\markboth{#1}{}} 
\lhead{} 
\chead{\textit{\thetitle}} 
\rhead{} 
\lfoot{} 
\cfoot{} 
\rfoot{\footnotesize Page \thepage\ of \pageref{LastPage}}
\fancypagestyle{firstpage}{ 
	\fancyhf{}
	\renewcommand{\footrulewidth}{0pt} 
}
\newcommand{\authorstyle}[1]{{\large\usefont{OT1}{phv}{b}{n}\color{DarkRed}#1}}
\newcommand{\institution}[1]{{\footnotesize\usefont{OT1}{phv}{m}{sl}\color{Black}#1}}
\usepackage{titling} 
\newcommand{\HorRule}{\color{DarkGoldenrod}\rule{\linewidth}{1pt}}
\pretitle{
	\vspace{-30pt} 
	\HorRule\vspace{10pt} 
	\fontsize{32}{36}\usefont{OT1}{phv}{b}{n}\selectfont 
	\color{DarkRed} 
}
\posttitle{\par\vskip 15pt} 
\preauthor{} 
\postauthor{ 
	\vspace{10pt} 
	\par\HorRule 
	\vspace{20pt} 
}
\usepackage{lettrine}
\usepackage{fix-cm}	
\newcommand{\initial}[1]{
	\lettrine[lines=3,findent=4pt,nindent=0pt]{
		\color{DarkGoldenrod}
		{#1}
	}{}
}
\usepackage{xstring}
\newcommand{\lettrineabstract}[1]{
	\StrLeft{#1}{1}[\firstletter] 
	\initial{\firstletter}\textbf{\StrGobbleLeft{#1}{1}}
}
\usepackage[backend=bibtex,style=authoryear,natbib=true]{biblatex}
\addbibresource{example.bib} 
\usepackage[autostyle=true]{csquotes}
 % Specifies the document structure and loads requires packages

\title{Implementando OpenGL con Oculus Rift SDK}

\author{
	\authorstyle{Jaime Margolin\textsuperscript{1}, Daniel Monzalalala\textsuperscript{1} and Juan Carlos León\textsuperscript{1}} % Authors
	\newline\newline 
	\textsuperscript{1}\institution{Instituto de Estudios Superiores del Tecnológico de Monterrey Campus Santa Fe}\\  }

\date{\today} 
\begin{document}

\maketitle 

\thispagestyle{firstpage} % Apply the page style for the first page (no headers and footers)

%----------------------------------------------------------------------------------------
%	ABSTRACT
%----------------------------------------------------------------------------------------

\lettrineabstract{El propósito del paper esta enfocado en dar una nueva perspectiva de como se puede incluir el uso de básicas y viejas funciones de OpenGL con la tecnología más sofisticada y moderna como lo es el SDK de Oculus y las librerías que utiliza esta última, y como estas tecnologías pueden ir de1 la mano para crear nuevas y mejores soluciones en el campo de las gráficas computacionales. Para esto se utilizaron librerías de modelos que el mismo SDK provee, logrando así una conjunción más armonica para el resultado final. Este paper ayudará a todos aquellos que en su momento quieran explorar el mundo de "VR" y quieran implementarlo de manera fácil y que quieran utilizar OpenGL envez del estandar para VR de DirctX.}

%----------------------------------------------------------------------------------------
%	ARTICLE CONTENTS
%----------------------------------------------------------------------------------------

\section{Introducción}

En la actualidad, hay campos que han surgido dentro del desarrollo de las tecnologías computacionales, siendo la Realidad Virtual uno de los que parece tener más futuro, el poder simular ambientes completos y poder hacer que el mismo usuario puede estar inmerso en el mismo, haciendole pensar que esta en un ambiente como en el que esta acostumbrado a interactuar.

Ahora una de las prácticas que se ha vuelto de las más comúnes es el uso de Oculus Rift, el cual es un dispositivo que puede simular en forma 3D ambiantes, los cuales son controlados por el mismo dispositivo, este dispositivo es un tanto parecido a unos lentes, haciendo asi que la combinacion entre los lentes y un buen equipo de sonido portátil (audífonos) ó estático (bocinas) puedan hacer que el usuario se sienta en el lugar que se esta simulando, un lugar que puede ver ante sus propios ojos e inclusive en algunas ocasiones y con algunos otros aditamentos, hacer que se pueda interactuar de manera \textbf{\textit{"normal"}}.



%----------------------------------------------------------------------------------------
\section{Trabajo Previo}

El trabajo previo en el campo del Oculus SDK esta desarrollado en DirectX y los mismos desarrolladores del SDK recomiendan utilizarla. Encontramos muy poco realizado con OpenGL en el campo del oculus rift....


\subsection{Paper 1}



\subsection{Paper 2}


\section{Desarrollo}
Nosotros intentamos crear una pecera  con distintos animales marinos utilisando el SDK del oculus rift  \textbf{DK2} para OpenGL. Por las funciones proveidas pro el SDK para OpenGL optamos por realizar este acuario con \textit{pixel art} que es una especie de representacion de los objetos en formas ams cuadradas al estilo de los videojeugos de \textbf{8 bit}.

El desarrollo consistio en tomar como base el (unico) proyecto ejemplo que tenia el SDK en OpenGL. Este ejemplo consistia de un pequeño cuarto con una mesa una silla un cubo que gira y un mueble de pared. lo que se realizo fue investigar como estaba el codigo compuesto. Se entonctro la funcion
\lstset{language=C++,
                keywordstyle=\color{blue},
                stringstyle=\color{red},
                commentstyle=\color{green},
                morecomment=[l][\color{magenta}]{\#}
}
\begin{lstlisting}
AddSolidColorBox(float x1, float y1,
 float z1, float x2, float y2, 
 float z2, DWORD c)
\end{lstlisting}
que lo que hace es agregar una "caja a la escena". esta funcion recibe como parametros la x, y, z iniciales , la x, y, z finales y el color de la caja en hexadecimal.

\subsection{armado del mundo}
\subsubsection{La pecera}

El mundo consta de 4 paredes 1 piso y 1 techo. estos forman la pecera en la que los distintos animales marinos estaran nadando.
Cada una de las paredes y el techo son creados agregando un cubo con 2 de sus aldos grandes y el 3 muy pequeño para que asi se asemeje a una pared delgada. El piso es creado de igual forma. por ultimo los colores que se le asiganaron a las paredes y al techo fue el mismo tono de azul, esto se eligio para lograr un efecto de continuidad. este efecto evita que se peuda diferenciar el momento en el que empiesa una pared y otra y donde conectan con el techo, asi dando un mayor sentido de profundidad.
Al piso se le asigno un color amarillento para que se parezca a la arena del mar. 
se intento agregar alguna de las texturas que ya estan incluidas en el SDK al piso para hcerlo parecer mas real pero las texturas del SDK daban la impresion de ser mosaicos y esto restaba un poco de realismo a la escena.  

\subsubsection{Las algas}
Las algas constan de un tallo de altura semi aleatoria. La altura del tallo es de 0.8 mas un numero aleatorio que va del $0.0$ al $0.7$ esto nos da una altura alteatoria de $0.8$ a $1.5$. Tambien se agregan hojas a la algas. Las hojas constan de 4 cubos pequeños acomodados en diagonal a partir del tallo. a una altura aleatoria de $0.2$ a $0.3$. cada alga puede tener de 0 a 4 hojas y esto tambien se decide de forma aleatoria. Y por ultimo la posicion de las hojas tambien se decide de forma aleatoria. Esto significa que sin importar la cnatidad de algas que se agreguen a la escena habra una gran variadad de algas con apariencias ligeramente diferentes para darle un toque de mayor realismo a la pecera.

\subsubsection{Los peces}
Los peces estan compuestos por 3 elementos principales: el cuerpo, la cara, las aletas. el cuerpo del pez consiste de un cubo grade de $1x1x1$. Los demas elementos del pez van "pegados" al cuerpo. la cara del pez esta compuesta por una boca que es un rectangulo pequeño en el centro en el eje $X$ y a $1/3$ del eje $Y$ del lado frontal del cubo, los ojos son 2 cuadrito negros a $1/3$ y $2/3$ del eje $X$ y a $2/3$ del eje $Y$ del lado frontal del pez. 

Las aletas entan divididas de la siguiente forma: 2 aletas laterales, 1 aleta dorsal y 1 aleta trasera. Las aletas laterales constan de 2 rectangulos delgados. uno mas pequeño que esta directamente pegado al pez y otro mas grande conectado al primero. los colores de las aletas son diferentes al del cuerpo del pez y cada uno de los rectangulos tiene su propio color.

La aleta dorsal esta compuesta de un rectangulo horizontal largo y de otro mas corto. esta aleta esta en la parte superior del pez. ambos rectangulos estan directamente pegados al cuerpo del pez, primero el rectangulo grande seguido del chico. los colores son los mismos que los de las aletas laterales. 

la aleta trasera, o cola, esta compuesta de 3 rectangulos de tamaño creciente y ancho decresiente. el mas pequeño y grueso esta pegado al pez mientras que el mas grande y delgado es el mas alejado. el primer rectangulo va pegado al pez de forma vertical, el segudo va pegado a este y el tercero al segundo. los colores de estos rectangulos son, el primero y tercero tienen el color del primer rectangulo de las aletas laterales y el 2 rectangulo el color del segundo rectangulo de las aletas laterales.

\subsubsection{tiburon}
El tiburon sigue la misma logica de contruccion que los peces unicamente se modifico el largo del cuerpo que para el tiburon es de $1x1x3$ y el grosor de las aletas, etas se hicieron mas gruesas para darle un toque de realismo. tambien se adecuo el posisionamiento de las aletas para que se mantuviera la proporcion del cuerpo. el tiburon es completamente negro con tonos grises en las aletas y ojos blancos.

\subsubsection{Erizo}
Para darle variedad a la vida marina en la pecera se decidio agregar un erizo de mar. el erizo tiene un cuerpo cuadrado de $0.5x0.5x0.5$ y 5 espinas. su cuerpo y espinas son de color gris. las espinas fueron creadas de forma en que todas estuvieran distibuidas por el cuepro del erizo y tuvieran longitudes ligeramente diferentes para dar un toque de realismo.

\subsection{Construccion del mundo}
Con todos los elementos del mundo armados se tienen que agregar a la escena. Para dar un toque de realismo se decidio que las posiciones de los peces como de las algas deberian ser aleatorias. 

\subsubsection{Tiburon y Erizo}
Por sus caracteristicas y funciones dentro de la escena se opto por dejar al tiburon y al erizo en una posicion de inicio fijas.

\subsubsection{Algas}
Las algas de decidio que deberia de tener un posicion aleatoria dentro de la pecera para dar un toque de realismo. cada vez que se corre el programa las algas apareserane n lugares distintos. Como se agrega una gran cantidad de algas a la escena el darles una posicion aleatoria ayuda a distribuirlas. las algas se crean en un rango de $-10$ a $10$ en $X$ y de $-20$ a $20$ en $Z$, esto para abarcar la totalidad de la pecera.

\subsubsection{Peces}
La posicion inicial de los peces tambien es aleatoria para dar un toque de realismo. Pero para mantenerlo visialmente estetico se decidio que la altura a la que nadan los peces seria pre definida, asi se evita que los peces pasen unos sobre otros. la altura mas baja es para los que nada en linea vertical (recorrido largo en el rectangulo que forma la pecera) mientras que los que nadan mas alto son los que recorren la pecera de forma horizontal (recorrido corto en el rectangulo que forma la pecera).

\subsection{Animaciones}
Para dar un toque de realismo se animaron todos los animales marinos de la escena. Las animaciones fueron de las partes mas retadoras ya que se trato de mantener una animacion congruente con el \textbf{pixel art} de la escena pero a la vez de hacerlo parecido a la realidad para mentener el toque de realismo que se le esta intentando dar al proyecto.

\subsubsection{Erizo}
El erizo es el animal con el movimiento mas siemple. Él se mueve siempre en un circulo alrededor del centro ($x/2,y/2$) de la pecera, tambien es el mas rapido de todos los animales y el único que se mueve por el piso.

\subsubsection{Tiburon}
A peticion del profesor \textit{"Estaria padre que agreguen un tiburon asi que vaya como acechando a los otros peces"} se agrego el tiburon y la animacion tenia que ir de acuerdo a dicha peticion. se opto por el movimiento en forma de \textit{8 extendido}, para que entre y salga de la escena como si estuviera acechando a los peces. El tiburon es el mas lento para dar un toque de realismo al acechar.

 \subsubsection{Peces}
Los peces son los animales con mas movimientos. para dar un toque de realismo se les programo un movimiento hacia el frente mientras que tienen un movimiento ondulatorio ya sea hacia arriba y abajo o de derecha a izquierda. 
 
Ademas de tener una posicion de origen aleatoria los peces tienen movimiento aleatorio, la velocidad con la que avanzan hacia el frente es aleatoria entre $0.22$ a $0.44$. esto hace que los peces se muevan a diferentes velocidades y da un toque de realismo a la escena.

Con fines demostrativos para la exposicion se decidio fijar el movimiento ondulatorio de los peces de la siguiente forma. lo que se mueven de forma vertical ondulan de arriba abajo mientras que lo que de mueve en forma horizontal ondulan de derecha a izquierda. 

\section{Complicaciones}
\subsection{figuras}
El primer problema con el que se encontro es que el SDK para el oculus únicamente tiene una figura, \textbf{el cubo}, por lo que la primera restriccion y la mas compleja fue el tener que armar todo con subos. Tambien la funcion que crea los cubos recibe como parametros la XYZ inicial y final, a diferencia de las funciones basicas que se vieron en clase que reciben unicamente el tamaño del cubo (y este es creado en la posicion en la que se encuentra GL en ese momento). Por lo que todas las posiciones y movimientos de la escena fueron calculados por nosotros. 

para lograr las distintas figuras de la pecera se hicieron muchos claculos para lograr modificar los cubos a que fueran cuadrados y rectangulos que pudieramos usar. ademas del calculo que hicimos para posicionarlos unos ocn respecto de otros.

\subsection{OpenGL SDK vs lo visto en clase}
Encontramos diferencias sustnaciales entre lo que se usa en el SDK del oculus y lo que se vio en clase. Aquí lsitaremos las que mas nos causaron problemas. \begin{itemize}
\item GL main loop - En este proyecto no hay un main loop que se paresca al main loop de GL. la funcion del main loop tiene funcionalidad similar pero no acepta ninguno de las funciones de GL dentro.
\item GL init - no hay una funcion de init como en GL.
\item funciones de GL - en el SDK del oculus no sirven ninguna de las siguientes funciones que se vieron en clase
\begin{itemize}
\item push matrix
\item pop matrix
\item rotate
\item scale 
\item translate
\item curvas 
\item superficies
\end{itemize}
\item luz - la luz como se vio en clase se agrega 1 (o varias) fuente de luz a la escena la cual interactua con los materiales de los objetos para ser reflejada. en el proeycto no hay ninguna fuente de luz, si no que cada objeto tiene su propio color, que de cierta forma es la luz que este objeto irradia, pero sin afectar a las demas o reflejarse en otros objetos.
\end{itemize}

\subsection{.OBJ's}
Se intento cargar archivos \textbf{.OBJ} de muchas maneras y ninugna permitio cargarlos. se probaron varias librerias y metodos entre los que destacan: glm (visto en clase) y assimp.

Inclusive se intento cargar los archivos con Dirctx y tampoco se pudo

\section{Resultados}

\section{Trabajo Futuro}

\section{Conclusiones}
%----------------------------------------------------------------------------------------
%	BIBLIOGRAPHY
%----------------------------------------------------------------------------------------

\printbibliography[{title = "example.bib" }] % Print the bibliography, section title in curly brackets

%----------------------------------------------------------------------------------------

\end{document}
